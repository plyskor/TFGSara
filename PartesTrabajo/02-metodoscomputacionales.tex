\chapter{Detalles computacionales}

Para el diseño de la estructura de todos los ácidos a estudiar se ha utilizado el programa Gaussian09, cuyo software resuelve la ecuación de Schrödinger basándose en la teoría de orbitales moleculares para obtener una serie de propiedades atómicas y moleculares.

Los cálculos DFT expuestos en este trabajo se han llevado acabo también con el programa Gaussian09, se ha utilizado el funcional B3LYP que es un método híbrido que contiene elementos DFT y la teoría Hartree-Fock, HF se utiliza para expresar la energía de intercambio en la aproximación del potencial de correlación de intercambio en la ecuación de kohn-Sham. Se realizaron optimizaciones de estructura completa utilizando la base 6-311G**, con la opción "opt+freq" como tipo de trabajo, donde todos los puntos estacionarios son mínimos verdaderos, no hay frecuencias imaginarias. Los datos de energía HF se calcularon utilizando la base 6-311+G(3df,2p) por su mayor precisión.

Las componentes termodinámicas se calcularon utilizando métodos estándar \cite {quimica4}, teniendo en cuenta las energías del punto cero (ZPE) y las correcciones de temperatura de la entalpía (TCE) y la energía libre de Gibbs (GFE). Al usar la base mencionada antes (6-311G**), estos números llevan una corrección, las energías escaladas serían las siguientes: \\
$$ H (0K) = E_{HF} + 0.9887*ZPE $$ 
$$ H (298K) = E_{HF} + TCE + ZPE - 0.9887*ZPE $$ 
$$G = E_{HF} + GFE $$ \\


