\chapter{Detalles computacionales}

La química computacional es una herramienta útil para estudiar los ácidos en fase gas. Es una técnica que hace uso de los fundamentos cuánticos para calcular las estructuras y las propiedades de moléculas en distintos estados. Sus resultados pueden predecir propiedades físico-químicas y es el complemento ideal que se ha revelado imprescindible para acompañar al experimento. Los estudios computacionales pueden llevarse a cabo con el fin de facilitar el trabajo en el laboratorio, ser usados para predecir moléculas hasta la fecha totalmente desconocidas o explotar mecanismos de reacción que no han sido fáciles de estudiar experimentalmente. Así, la química computacional puede ayudar a la experimental a predecir objetos químicos totalmente nuevos y explicar fenómenos experimentales, la idea de las herramientas de simulación es tratar de ir hacia donde las demás técnicas no logran llegar o les cuesta mucho. {\bfseries Para ello se hace uso de equipos muy rápidos llamados clusters, son una serie de procesadores que están interconectados entre sí permitiendo hacer cálculos en paralelo, es decir, hacer un cálculo en muchos procesadores simultáneamente. NO SE SI PONER ESTO}

La energía total está determinada por una solución aproximada de la ecuación de Schrödinger dependiente del tiempo, haciendo uso de la aproximación de Born-Oppenheimer se simplifica dicha ecuación. Esto conduce a la evaluación de la energía total como una suma de energía electrónica en las posiciones fijas del núcleo más la energía de repulsión del mismo. Algunos métodos para determinar la energía total y predecir la estructura molecular de un sistema son, entre otros muchos: HF (hartree Fock), MPx (Moller-Plesset), DFT (teoría del funcional de densidad), etc. \cite {quimica2}

Los métodos ab initio como Hartree-Fock ayudan a resolver la ecuación de Schrödinger asociada al Hamiltoniano molecular, esta resolución es aproximada, conociendo siempre el margen de error de antemano. Estos métodos no incluyen ningún parámetro empírico en sus ecuaciones, éstas vienen directamente de principios teóricos sin incluir datos experimentales, tampoco incluye la correlación electrónica por lo que el error de dichos métodos es considerable.

La teoría del funcional de densidad es utilizada para determinar la estructura electrónica molecular, expresan la energía total en términos de densidad total en lugar de la función de onda, en estos cálculos hay un hamiltoniano aproximado y una expresión aproximada para la densidad electrónica total. La calidad de los resultados obtenidos, y su eficiencia computacional hacen que los métodos DFT se utilicen con frecuencia, la precisión de éstos se debe a que se incluye la correlación electrónica, aunque de forma aproximada ya que el funcional no es exacto, se ha probado su eficiencia en muchos sistemas orgánicos e inorgánicos.

Por otro lado los métodos compuestos G1, G2, G3 y G4 proporcionan una mayor precisión en los resultados, puesto que están basados en la teoría del funcional de densidad, combinando también métodos como Hartree-Fock, Moller-Plesset.

Todos estos métodos también son preparados para la imaginación de nuevos compuestos. Nuestro estudio se basa en la estabilidad de isómeros y confórmeros  en fase gaseosa

Para el diseño de la estructura de todos los ácidos a estudiar se ha utilizado el programa Gaussian09, cuyo software resuelve la ecuación de Schrödinger basándose en la teoría de orbitales moleculares para obtener una serie de propiedades atómicas y moleculares.

Los cálculos DFT expuestos en este trabajo se han llevado acabo también con el programa Gaussian09, se ha utilizado el funcional B3LYP que es un método híbrido que contiene elementos DFT y la teoría Hartree-Fock, HF se utiliza para expresar la energía de intercambio en la aproximación del potencial de correlación de intercambio en la ecuación de kohn-Sham. Se realizaron optimizaciones de estructura completa utilizando la base 6-311G**, con la opción "opt+freq" como tipo de trabajo, donde ningún compuesto tiene frecuencias imaginarias en la geometría optimizada final. Los datos de energía total Hartree-Fock se calcularon utilizando la base 6-311+G(2df,2p) por su mayor precisión.


Las componentes termodinámicas se calcularon utilizando métodos estándar \cite {quimica4}, teniendo en cuenta las energías del punto cero (ZPE) y las correcciones del punto cero y térmicas de la entalpía (TCE) y la energía libre de Gibbs (GFE). Al usar la base mencionada antes (6-311G**), estos números llevan una corrección, las energías escaladas serían las siguientes: \\
$$ H (0K) = E_{HF} + 0.9887*ZPE $$ 
$$ H (298K) = E_{HF} + TCE + ZPE - 0.9887*ZPE $$ 
$$G = E_{HF} + GFE $$

Se consideró la conformación de energía más baja de cada compuesto en el estudio de estabilidad y se han diseñado nuevas estructuras para hacer un análisis más extenso de la acidez en fase gas.

Se realizó un análisis de poblaciones NBO, éste se basa en un método para transformar de manera óptima una función de onda en forma localizada, correspondiente a los elementos de un centro (pares solitarios-LP), y de dos centros (enlaces-BD) tal como se observarían en la estructura de Lewis de una molécula. En dicho análisis, el conjunto de bases orbitales atómicas a la entrada se transforma a través de orbitales atómicos naturales (NAOs) y orbitales híbridos (NHOs) en orbitales de enlace natural (NBOs). Los NBO obtenidos corresponden a la imagen normalmente utilizadda de Lewis donde los enlaces de dos centros y electrones no enlazantes estan localizados \cite{quimica5}.

En este estudio teórico se utilizó el método desarrollado por Bader, AIM \cite{quimica6} con el fin de tener evidencias teóricas de las interacciones por formaciones de puentes de hidrógeno, basándose en la densidad electrónica, los puntos críticos de densidad electrónica fueron localizados con el programa Gaussian.

Se uso el método compuesto G4 para una mayor precisión de resultados, previamente dudosos, con el fin de compararlos con los experimentales.
 