\chapter{Metodología}

La química computacional es una herramienta útil para estudiar los ácidos en fase gas. Es una técnica que hace uso de los fundamentos cuánticos para calcular las estructuras y las propiedades de moléculas en distintos estados. Sus resultados pueden predecir propiedades físico-químicas y es el complemento ideal que se ha revelado imprescindible para acompañar al experimento. Los estudios computacionales pueden llevarse a cabo con el fin de facilitar el trabajo en el laboratorio, ser usados para predecir moléculas hasta la fecha totalmente desconocidas o explotar mecanismos de reacción que no han sido fáciles de estudiar experimentalmente. Así, la química computacional puede ayudar a la experimental a predecir objetos químicos totalmente nuevos y explicar fenómenos experimentales, la idea de las herramientas de simulación es tratar de ir hacia donde las demás técnicas no logran llegar o les cuesta mucho, con la revolución tecnológica hoy en día, se pueden tratar sistemas muy complejos.

La energía total está determinada por una solución aproximada de la ecuación de Schrödinger dependiente del tiempo, haciendo uso de la aproximación de Born-Oppenheimer se simplifica dicha ecuación. Esto conduce a la evaluación de la energía total como una suma de energía electrónica en las posiciones fijas del núcleo más la energía de repulsión del mismo. Algunos métodos para determinar la energía total y predecir la estructura molecular de un sistema son, entre otros muchos: HF (hartree Fock), MPx (Moller-Plesset), DFT (teoría del funcional de densidad), etc.

Los métodos ab initio como Hartree-Fock ayudan a resolver la ecuación de Schrödinger asociada al Hamiltoniano molecular, esta resolución es aproximada, conociendo siempre el margen de error de antemano. Estos métodos no incluyen ningún parámetro empírico en sus ecuaciones, éstas vienen directamente de principios teóricos sin incluir datos experimentales, tampoco incluye la correlación electrónica por lo que el error de dichos métodos es considerable.\cite {quimica2}

La teoría del funcional de densidad es utilizada para determinar la estructura electrónica molecular, expresan la energía total en términos de densidad total en lugar de la función de onda, en estos cálculos hay un hamiltoniano aproximado y una expresión aproximada para la densidad electrónica total. La calidad de los resultados obtenidos, y su eficiencia computacional hacen que los métodos DFT se utilicen con frecuencia, la precisión de éstos se debe a que se incluye la correlación electrónica, aunque de forma aproximada ya que el funcional no es exacto, se ha probado su eficiencia en muchos sistemas orgánicos e inorgánicos. \cite{quimica9} \cite {quimica2}

Por otro lado los métodos compuestos G1, G2, G3 y G4 proporcionan una mayor precisión en los resultados, puesto que están basados en la teoría del funcional de densidad, combinando también métodos como Hartree-Fock, Moller-Plesset.
Todos estos métodos también son preparados para la imaginación de nuevos compuestos.

Para el diseño de la estructura de todos los ácidos a estudiar se ha utilizado Gaussian09 \cite{Gaussian}, cuyo programa también ha ayudado a resolver los cálculos DFT expuestos en este trabajo, se ha utilizado el funcional B3LYP \cite{quimica10} \cite{quimica9} que es un método híbrido que contiene elementos DFT y la teoría Hartree-Fock, HF se utiliza para expresar la energía de intercambio en la aproximación del potencial de correlación de intercambio en la ecuación de kohn-Sham. Se realizaron optimizaciones de estructura completa utilizando la base 6-311G**, donde ningún compuesto tiene frecuencias imaginarias en la geometría optimizada final. Los datos de energía total Hartree-Fock se calcularon utilizando la base 6-311+G(2df,2p) por su mayor precisión.


Las componentes termodinámicas se calcularon utilizando métodos estándar \cite {quimica4}, teniendo en cuenta las energías del punto cero (ZPE) y las correcciones del punto cero y térmicas de la entalpía (TCE) y la energía libre de Gibbs (GFE). Al usar la base mencionada antes (6-311G**), estos números llevan una corrección, las energías escaladas serían las siguientes: \\
$$ H (0K) = E_{HF} + 0.9887*ZPE $$ 
$$ H (298K) = E_{HF} + TCE + ZPE - 0.9887*ZPE $$ 
$$G = E_{HF} + GFE $$

Se consideró la conformación de energía más baja de cada compuesto en el estudio de estabilidad y se han diseñado nuevas estructuras para hacer un análisis más extenso de la acidez en fase gas.

Se realizó un análisis de poblaciones NBO (orbitales de enlace natural), éste se basa en un método para transformar de manera óptima una función de onda en forma localizada, correspondiente a los elementos de un centro (pares solitarios-LP), y de dos centros (enlaces-BD) tal como se observarían en la estructura de Lewis de una molécula. En dicho análisis, el conjunto de bases orbitales atómicas a la entrada se transforma a través de orbitales atómicos naturales (NAOs) y orbitales híbridos (NHOs) en orbitales de enlace natural (NBOs). Los NBO obtenidos corresponden a la imagen normalmente utilizada de Lewis donde los enlaces de dos centros y los electrones no enlazantes estan localizados \cite{quimica5} \cite{quimica8}.

En este estudio teórico se utilizó el método desarrollado por Bader, AIM \cite{quimica6}, este es un postulado de mecánica cuántica relativo a o que todo lo que puede conocerse respecto a un sistema está contenido en su función de estado $\Psi$. El valor de una cantidad física se obtiene de la función de estado mediante la acción del correspondiente operador en $\Psi$. De esta forma, a partir de la función de estado, puede obtenerse más información química que simplemente las energías y sus geometrías asociadas.
Se localizaron los máximos de densidad de las moléculas con el fin de mostrar el efecto de los halógenos y de la hidratación específica en la acidez.

Se uso el método compuesto G4 para una mayor precisión de resultados, previamente dudosos, con el fin de compararlos con los experimentales. \cite{quimica11}
 