\chapter{Conclusión}

\begin{enumerate}
	\item Para determinar la estabilidad de diferentes confórmeros de ácidos en fase gas, hay que tener en cuenta la posible formación de enlaces de hidrógeno intramoleculares.
	\item La estabilidad de oxácidos en fase gas depende de la electronegatividad de los sustituyentes y de el tamaño de los mismos, siendo dentro de un período, más estable una molécula con los sustituyentes más electronegativos, y dentro de un grupo más estable la molécula con los sustituyentes de mayor tamaño.
	\item Gracias a un análisis NBO,se muestra como efecto inductivo en los ácidos tiene un papel de elevada importancia en la acidez de éstos en fase gas, utilizando métodos DFT con una base relativamente grande como es la 6-311+G(2df,2p) se tiene la sucesión indicada en el apartado anterior y produce un acuerdo bastante satisfactorio con los valores experimentales,  {\bfseries\textcolor{red} {aun así sobreestima la acidez de algunos ácidos alrededor de 5 Kcal/mol.}}, los cálculos G4 son mucho más precisos y sería el siguiente paso para continuar con el estudio.
	\item La acidez aumenta con el efecto de una hidratación específica en fase gas por la formación de enlaces de hidrógeno muy fuertes, pero el impacto de esta hidratación compite con el efecto inductivo del halógeno y el incremento de acidez no es muy elevado.
	\item Los cálculos DFT se completaron en 19 moléculas sin mediciones experimentales, proporcionando así una base de datos termoquímica y estructural rigurosa para futuros estudios termodinámicos.
	
	
\end{enumerate}