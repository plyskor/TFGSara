\begin{thebibliography}{9}

\bibitem{quimica1} 
Alcañiz, E. de Jesús
\textit{Química Inorgánica}. 
Recuperado de http://www3.uah.es Págs. 35-54
2012

\bibitem{quimica7}
Jones, L. / Atkins (2012)
\textit{Principios de Química. 5a ed.} Editorial Médica Panamericana
Págs. 401-453

\bibitem{quimica10}
C. Lee, W. Yang and R.G. Parr.
\textit{ Phys. Rev. B 37 (1988) 785}

\bibitem{quimica9}
A. D. Becke
\textit{J. Phys. Chem., 88 (1998) 1053}

\bibitem{quimica3} 
\textit{J. Am. Chem. Soc., Vol. 122, No. 21, }
2000 Págs. 5114-5124

\bibitem{quimica2} 
Bertran Rusca, J.; Branchadell Gallo, V.; Moreno Ferrer, M.; Sodupe Roure,M. (2012)
\textit{Química Cuántica. 1a ed.} Editorial Síntesis
Págs. 169-198

\bibitem{quimica4}
Joseph W. Ochterski, Ph.D
\textit{Thermochemistry in Gaussian}.
2000

\bibitem{quimica8}
Weinhold, F; Landis, C. R. (2005) 
\textit{Valency and Bonding. A natural Bond Orbital Donor-Acceptor Perspective.}
Cambridge

\bibitem{quimica6}
Bader, R. F. (1990). Atoms in molecules. John Wiley., Sons, Ltd.

\bibitem{quimica11}
Curtiss, L. A., Redfern, P. C., Raghavachari, K. (2007). Gaussian-4 theory. The Journal of chemical physics, 126(8), 084108.
ISO 690	


\bibitem{AIM}
Bader, R. F. W. (2000). AIM 2000 program. McMaster University, Hamilton, ON.

\bibitem{Gaussian}
Frisch, A. (Ed.). (2004). Gaussian 03. Gaussian.

\bibitem{47}
Bickelhaupt, F. M., Baerends, E. J. (2007). Kohn‐Sham Density Functional Theory: Predicting and Understanding Chemistry. Reviews in Computational Chemistry, Volume 15, 1-86.
\end{thebibliography}
