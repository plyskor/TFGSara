\begin{thebibliography}{9}

\bibitem{quimica1} 
Alcañiz, E. de Jesús
\textit{Química Inorgánica}. 
Recuperado de http://www3.uah.es Págs. 35-54
2012

\bibitem{quimica7}
Jones, L. / Atkins (2012)
\textit{Principios de Química. 5a ed.} Editorial Médica Panamericana
Págs. 401-453

\bibitem{quimica10}
C. Lee, W. Yang and R.G. Parr.
\textit{ Phys. Rev. B 37 (1988) 785}

\bibitem{quimica9}
A. D. Becke
\textit{J. Phys. Chem., 88 (1998) 1053}

\bibitem{quimica3} 
\textit{J. Am. Chem. Soc., Vol. 122, No. 21, }
2000 Págs. 5114-5124

\bibitem{quimica2} 
Bertran Rusca, J.; Branchadell Gallo, V.; Moreno Ferrer, M.; Sodupe Roure,M. (2012)
\textit{Química Cuántica. 1a ed.} Editorial Síntesis
Págs. 169-198

\bibitem{quimica4}
Joseph W. Ochterski, Ph.D
\textit{Thermochemistry in Gaussian}.
2000

\bibitem{quimica5}
Frank Weinhold and Eric D. Glendening. (s.f) Natural Bond Orbital. Analysis Programs. Madison, Wisconsin 53706, USA: 
\textit {Natural Bond Orbital 6.0 Homepage}. Recuperado de $http://nbo.chem.wisc.edu/nbo6ab_man.pdf$

\bibitem{quimica8}
Weinhold, F; Landis, C. R. (2005) 
\textit{Valency and Bonding. A natural Bond Orbital Donor-Acceptor Perspective.}
Cambridge

\bibitem{quimica6}
Bader, R. F. W
\textit{Acc. Chem. Res., 1985, 18 (1), pp 9–15}

\bibitem{quimica11}
J. Chem. Eng. Data 2010, 55, 5359–5364

\bibitem{AIM}
Bader, R. F. W. (2000). AIM 2000 program. McMaster University, Hamilton, ON.

\bibitem{Gaussian}
Frisch, A. (Ed.). (2004). Gaussian 03. Gaussian.

\end{thebibliography}
