
\chapter*{Abstract} 


   bla bla bla ingles

En este ensayo se trata la estabilidad estructural de ácidos en fase gas, y la variación de éstos al sustituir un átomo electropositivo (hidrógeno) por uno electronegativo (halógeno) como el flúor y el cloro. El principal tema abordado es el estudio de la acidez intrínseca de una serie de oxácidos y ácidos orgánicos, siendo este apartado de gran interés, ya que la ausencia de interacciones con un disolvente resulta a una acidez muy diferente.
 Los temas de estudio son las diferencias de acidez de dichos ácidos, los cambios que se producen en la acidez intrínseca con la sustitución de hidrógenos por halógenos y la variación de dicha acidez cuando hidratamos de forma específica el ácido. El aumento de acidez observado al tener átomos más electronegativos en la molécula se ha hayado mediante el uso de distintos métodos teóricos. Se han llevado acabo cálculos DFT con el fin de calcular valores teóricos de la acidez de las moléculas a estudiar, También se ha utilizado el método NBO de análisis de poblaciones para describir los cambios en las configuraciones electrónicas de las moléculas y que son responsables del cambio de la acidez de éstas. Se han empleado cálculos G4 para hacer un análisis más preciso de algunos resultados dudosos con el fin de relacionar los resultados expuestos en el presente trabajo, es importante destacar que dichos resultados expuestos en este trabajo están de acuerdo con los datos experimentales previamente dados.


\vspace{5mm}
\textbf{Key words:} bla bla bla