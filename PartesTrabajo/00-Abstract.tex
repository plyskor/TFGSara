
\chapter*{Abstract} 

\begin{comment}
En este ensayo se trata la estabilidad estructural de ácidos clásicos en fase gas $H_3PO_4$, $H_2SO_4$ y $H_2CO_3$, y de ácidos cuando se sustituye un átomo electropositivo (hidrógeno) por uno electronegativo (halógeno) como el flúor y el cloro, $HFSO_4$, $HClCO_3$, $H_2FPO_4$, $HCl_2PO_4$ utilizando el método DFT B3LYP/6-311+G(2p,2f).

Con este método y haciendo un análisis de poblaciones NBO para describir los cambios en las configurciones electrónicas de las moléculas, se abordó también el estudio de la acidez intrínseca de una serie de oxácidos y ácidos orgánicos, siendo este apartado de gran interés, ya que la ausencia de interacciones con un disolvente resulta a una acidez muy diferente. 

La variación de acidez intrínseca según los sustituyentes de la molécula también se aborda en el presente proyecto, aumentando a medida que hay más sustituyentes electronegativos. La adición de una molécula de agua en el punto activo de la molécula se traduce en un aumento de su acidez.

Se han empleado cálculos G4 para hacer un análisis más preciso de algunos resultados dudosos con el fin de relacionar los resultados expuestos en el presente trabajo, es importante destacar que dichos resultados están de acuerdo con los datos experimentales previamente dados.
\end{comment}


In this essay we are treating about classic acids' structural stability in gas phase ($H_3PO_4$, $H_2SO_4$ and $H_2CO_3$), and also other acids after replacing an electropositive atom (hydrogen) with an electronegative one (halogen) such as fluorine or chlorine ($HFSO_4$, $HClCO_3$, $H_2FPO_4$, $HCl_2PO_4$), all of this using DFT B3LYP/6-311+G(2p,2f) method.

With this method (and also performing a natural bond orbital analysis (NBO) for describing the changes in molecules' electron configurations) we dealt with the study of intrinsic acidity of a series of oxoacids and organic acids. This is indeed a really interesting part of this work because the lack of interactions with a solvent leads to a much different acidity.

The variation of intrinsic acidity according to molecule's substituents is also treated in this document, increasing as we add more electronegative substituents.  The addition of a water molecule in the molecule's active point turns into increasing the acidity.

G4 calculations have been performed in order to make a more precise analysis about some unlikely results with the purpose of relating the resuslts exposed in this essay. It is important to emphasize that those particular results agree with the experimental data previously given to us.


\vspace{5mm}
\textbf{Key words:} intrinsic acidity, specific hydration.