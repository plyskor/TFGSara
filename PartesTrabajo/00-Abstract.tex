
\chapter*{Abstract} 


En este ensayo se trata la estabilidad estructural de ácidos clásicos en fase gas $H_3PO_4$, $H_2SO_4$ y $H_2CO_3$, y de ácidos cuando se sustituye un átomo electropositivo (hidrógeno) por uno electronegativo (halógeno) como el flúor y el cloro, $HFSO_4$, $HClCO_3$, $H_2FPO_4$, $HCl_2PO_4$ utilizando el método DFT B3LYP/6-311+G(2p,2f).
Con este método y haciendo un análisis de poblaciones NBO para describir los cambios en las configurciones electrónicas de las moléculas, se abordó también el estudio de la acidez intrínseca de una serie de oxácidos y ácidos orgánicos, siendo este apartado de gran interés, ya que la ausencia de interacciones con un disolvente resulta a una acidez muy diferente. La variación de acidez intrínseca según los sustituyentes de la molécula también se aborda en el presente proyecto, aumentando a medida que hay más sustituyentes electronegativos, la adición de una molécula de agua en el punto activo de la molécula se traduce en un aumento de su acidez, se utilizan cálculos 
Se han empleado cálculos G4 para hacer un análisis más preciso de algunos resultados dudosos con el fin de relacionar los resultados expuestos en el presente trabajo, es importante destacar que dichos resultados expuestos en este trabajo están de acuerdo con los datos experimentales previamente dados.


\vspace{5mm}
\textbf{Key words:} acidez intrínseca, hidratación específica.