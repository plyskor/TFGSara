\chapter{Introducción}

Para abordar el tema de la estabilidad y la acidez de los ácidos hay que tener en cuenta la definición ácido-base de Brönsted y de Lewis. \cite{quimica1}

Según Brönsted  \cite{quimica7} los ácidos son sustancias que actúan como dadoras de protones y las bases son especies que actúan como aceptoras. En esta definición, el término protón se refiere al ion de hidrógeno $H^+$, por lo que un ácido es una especie que contiene al menos un átomo de hidrógeno ácido que puede transferirse como un protón, a una especie que actúa como base. La teoría de Brönsted se centra en la transferencia de un protón de una especie a otra, Sin embargo los conceptos de ácidos y bases tienen un significado mucho más amplio que la transferencia de protones. Pueden clasificarse más sustancias como ácidos o bases si se consideran las definiciones desarrolladas por G. N. Lewis \cite{quimica7}, el cual dice que un ácido de Lewis es un aceptor de un par de electrones y una base de Lewis es un donante de un par de electrones. Cuando una base de Lewis dona un par de electrones a un ácido de Lewis, ambos forman en enlace covalente coordinado. Un protón ($H^+$) es un aceptor de un par de electrones y, por tanto, un ácido de Lewis, ya que puede aceptar un par solitario de electrones de una base de Lewis. es decir, un ácido de Brönsted es donante de un ácido de Lewis particular, un protón.

En dichas definiciones, ya que no se hace mención a ningún disolvente, se aplica a disolventes no acuosos e incluso a reacciones en fase gaseosa. El comportamiento ácido-base de una sustancia puede variar en gran medida con el disolvente. Por ejemplo, los ácidos fuertes en agua no existen como tal, sólo en su forma disociada y son más fuertes que el catión oxonio, puesto que la donación de protones ocurre desde el ácido al agua, la fortaleza de un ácido fuerte en agua se nivela a la fuerza del ion oxonio y la de cualquier base fuerte a la fuerza del $OH ^-$. Los ácidos y bases que se encuentren dentro del rango de estabilidad ácido-base en agua son débiles.

La adicez de una sustancia en fase gaseosa está relacionada con la energía necesaria para romper heterolíticamente el enlace H-A, esta energía se puede obtener como la suma de la energía necesaria para romper el enlace H-A homolíticamente más la energía necesaria para ionizar H y A (que es la suma del potencial de ionización del hidrógeno y la afinidad electrónica de A). Estos dos procesos son endotérmicos, por lo que la disociación de un ácido no es favorable termodinámicamente en fase gas \cite {quimica1}. En contraste, en disolución acuosa, los iones se hidratan fuertemente, sobretodo el protón por su pequeño tamaño, la energía en dicho proceso es la que explica la disociación de los ácidos en agua.

La cuestión abordada en el presente trabajo es la estabilidad estructural de isómeros y confórmeros de diversas estructuras en fase gas, se harán "puzzles" haciendo uso de herramientas computacionales  con el fin de determinar la estructura más estable, a medida que avancemos nos iran surguiendo diferentes preguntas; ¿cómo afectan dichos cambios a la acidez de las moléculas?, ¿cómo afectaría a esta propiedad una hidratación en un punto acídico?.

Con ayuda de las componentes termodinámicas de ácidos orgánicos y oxácidos, se hará un estudio riguroso del efecto que produce en la acidez, la sustitución de hidrógenos por halógenos en fase gaseosa, y una hidratación específica en el punto acidico. \cite{quimica3} 
