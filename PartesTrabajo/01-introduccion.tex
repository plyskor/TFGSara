\chapter{Introducción}
Para abordar el tema de la estabilidad y la acidez de los ácidos hay que tener en cuenta la definición ácido-base de Brönsted y de Lewis. Hola

Según Brönsted los ácidos son sustancias que actúan como dadoras de protones y las bases son especies que actúan como aceptoras. En esta definición, el término protón se refiere al ion de hidrógeno $H^+$, por lo que un ácido es una especie que contiene al menos un átomo de hidrógeno ácido que puede transferirse como núcleo, un protón, a una especie que actúa como base. La teoría de Brönsted se centra en la transferencia de un protón de una especie a otra, Sin embargo los conceptos de ácidos y bases tienen un significado mucho más amplio que la transferencia de protones. Pueden clasificarse más sustancias como ácidos o bases si se consideran las definiciones desarrolladas por G. N. Lewis, el cual dice que un ácido de Lewis es un aceptor de un par de electrones y una base de Lewis es un donante de un par de electrones. Cuando una base de Lewis dona un par de electrones a un ácido de Lewis, ambos forman en enlace covalente coordinado. Un protón ($H^+$) es un aceptor de un par de electrones y, por tanto, un ácido de Lewis, ya que puede aceptar un par solitario de electrones de una base de Lewis. es decir, un ácido de Brönsted es donante de un ácido de Lewis particular, un protón. 

En dichas definiciones, ya que no se hace mención a ningún disolvente, se aplica a disolventes no acuosos e incluso a reacciones en fase gaseosa. El comportamiento ácido-base de una sustancia puede variar en gran medida con el disolvente. Por ejemplo, los ácidos fuertes en agua no existen como tal, sólo en su forma disociada y son más fuertes que el catión oxonio, puesto que la donación de protones ocurre desde el ácido al agua, la fortaleza de un ácido fuerte en agua se nivela a la fuerza del ion oxonio y la de cualquier base fuerte a la fuerza del $OH ^-$. Los ácidos y bases que se encuentren dentro del rango de estabilidad ácido-base en agua son débiles. \cite {quimica1}

La adicez de una sustancia en fase gaseosa está relacionada con la energía necesaria para romper heterolíticamente el enlace H-A, esta energía se puede obtener como la suma de la energía necesaria para romper el enlace H-A homolíticamente más la energía necesaria para ionizar H y A (que es la suma del potencial de ionización del hidrógeno y la afinidad electrónica de A). Estos dos procesos son endotérmicos, por lo que la disociación de un ácido no es favorable termodinámicamente en fase gas. En contraste, en disolución acuosa, los iones se hidratan fuertemente, sobretodo el protón por su pequeño tamaño, la energía en dicho proceso es la que explica la disociación de los ácidos en agua.

La química computacional es una herramienta útil para estudiar los ácidos en fase gas. Es una técnica que hace uso de la química teórica para calcular las estructuras y las propiedades de moléculas en distintos estados. Sus resultados pueden predecir propiedades físico-químicas y es el complemento ideal que se ha revelado imprescindible para acompañar al experimento. Los estudios computacionales pueden llevarse a cabo con el fin de facilitar el trabajo en el laboratorio, ser usados para predecir moléculas hasta la fecha totalmente desconocidas o explotar mecanismos de reacción que no han sido fáciles de estudiar experimentalmente. Así, la química computacional puede ayudar a la experimental a predecir objetos químicos totalmente nuevos y explicar fenómenos experimentales, la idea de las herramientas de simulación es tratar de ir hacia donde las demás técnicas no logran llegar o les cuesta mucho. Para ello se hace uso de equipos muy rápidos llamados clusters, son una serie de procesadores que están interconectados entre sí permitiendo hacer cálculos en paralelo, es decir, hacer un cálculo en muchos procesadores simultáneamente.

La energía total está determinada por una solución aproximada de la ecuación de Schrödinger dependiente del tiempo, haciendo uso de la aproximación de Born-Oppenheimer se simplifica dicha ecuación. Esto conduce a la evaluación de la energía total como una suma de energía electrónica en las posiciones fijas del núcleo más la energía de repulsión del mismo. Algunos métodos para determinar la energía total y predecir la estructura molecular de un sistema son, entre otros muchos: HF (hartree Fock), MPx (Moller-Plesset), DFT (teoría del funcional de densidad), etc. \cite {quimica2}

Los métodos ab initio como Hartree-Fock ayudan a resolver la ecuación de Schrödinger asociada al Hamiltoniano molecular, esta resolución es aproximada, conociendo siempre el margen de error de antemano. Estos métodos no incluyen ningún parámetro empírico en sus ecuaciones, éstas vienen directamente de principios teóricos sin incluir datos experimentales.

La teoría del funcional de densidad es utilizada para determinar la estructura electrónica molecular, expresan la energía total en términos de densidad total en lugar de la función de onda, en estos cálculos hay un hamiltoniano aproximado y una expresión aproximada para la densidad electrónica total. Los métodos DFT son muy precisos y con un bajo coste computacional.

El análisis de los orbitales de enlace natural, análisis de poblaciones, es uno de los métodos utilizados para analizar la densidad electrónica basado en el análisis de la función de onda.

Por otro lado los métodos compuestos G1, G2, G3 y G4 proporcionan una mayor precisión en los resultados, puesto que están basados en la teoría del funcional de densidad, combinan también métodos como Hartree-Fock, Moller-Plesset.

Todos estos métodos también son preparados para la imaginación de nuevos compuestos. Es interesante el estudio de componentes termodinámicas de ácidos orgánicos y oxácidos como el $H_2SO_4, HPO_3, HNO_3...$, etc., la sustitución de sus hidrógenos por halógenos en fase gaseosa y el estudio de una hidratacón específica viendo el aumento o disminución de su acidez \cite{quimica3}, para así realizar dicho estudio de forma rigurosa realizando los cálculos anteriormente citados.
